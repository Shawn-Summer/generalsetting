% 文档类:ctexart(中文支持),A4 纸,12号字体
\documentclass[12pt,a4paper]{ctexart}

% 加载常用宏包(满足公式、绘图、引用等需求)
\usepackage{amsmath,amssymb}  % 数学公式支持
\usepackage{graphicx}         % 图片插入(可选)
\usepackage{tikz}             % 绘图支持(你的配置已开启语法高亮)
\usetikzlibrary{shapes}       % 新增:加载形状库,支持 diamond(菱形)等扩展形状
\usepackage{hyperref}         % 交叉引用、链接高亮
\usepackage{enumitem}         % 列表样式优化

% 参考文献数据库(测试用,实际可单独放在 .bib 文件中)
\begin{filecontents}[overwrite]{ref.bib}
@article{einstein1905,
  title={Zur Elektrodynamik bewegter K{\"o}rper},
  author={Einstein, Albert},
  journal={Annalen der Physik},
  volume={322},
  number={10},
  pages={891--921},
  year={1905},
  publisher={Wiley-VCH}
}

@book{latex2e,
  title={LaTeX2e 完全学习手册},
  author={刘海洋},
  publisher={清华大学出版社},
  year={2020},
  edition={5}
}
\end{filecontents}

% 文档标题、作者、日期
\title{VimTeX 配置测试文档}
\author{测试用户}
\date{\today}  % \today 自动显示当前日期,也可手动写死(如 2025年12月14日)

\begin{document}

% 生成标题(自动包含在目录中)
\maketitle

% 生成目录(测试 latexmk 自动更新目录功能)
\tableofcontents
\newpage  % 分页,让目录单独一页

% 1. 一级标题(测试目录、交叉引用)
\section{引言}
这是一份用于测试 VimTeX + XeLaTeX 配置的中文文档,包含以下核心功能:
\begin{itemize}[label=✓, itemsep=0.5em]
  \item 中文显示(依赖 XeLaTeX 编译器)
  \item 数学公式(行内公式和独立公式)
  \item 列表环境(有序/无序列表)
  \item 目录生成与更新
  \item 交叉引用(公式、章节、文献)
  \item TikZ 绘图(语法高亮已配置)
  \item 参考文献引用与生成
\end{itemize}

通过 \hyperref[sec:formula]{公式部分} 和 \hyperref[sec:tikz]{绘图部分} 可测试双向搜索功能:在 Neovim 光标定位到对应内容,按 \verb|<leader>lv| 可跳转到 PDF 对应位置;在 Skim 点击对应内容,可跳回 Neovim 源码。

% 2. 二级标题(测试多级目录)
\section{数学公式测试}
\label{sec:formula}  % 给章节添加标签,用于交叉引用

% 行内公式(测试正向/反向搜索精准度)
行内公式示例:质能方程 $E=mc^2$(来自 \cite{einstein1905}),勾股定理 $a^2 + b^2 = c^2$。

% 独立公式(带编号,支持引用)
\begin{equation}
  \label{eq:integral}
  \int_{-\infty}^{+\infty} e^{-x^2} dx = \sqrt{\pi}
\end{equation}
如公式 \ref{eq:integral} 所示,这是高斯积分的结果,常用于概率论和量子力学中。

另一个矩阵公式示例:
\begin{equation}
  \begin{pmatrix}
    1 & 2 & 3 \\
    4 & 5 & 6 \\
    7 & 8 & 9
  \end{pmatrix}
  \times
  \begin{pmatrix}
    x \\ y \\ z
  \end{pmatrix}
  =
  \begin{pmatrix}
    x+2y+3z \\ 4x+5y+6z \\ 7x+8y+9z
  \end{pmatrix}
\end{equation}

% 3. TikZ 绘图测试(你的配置已开启语法高亮)
\section{ TikZ 绘图测试}
\label{sec:tikz}

以下是用 TikZ 绘制的流程图(测试语法高亮和编译兼容性):
\begin{tikzpicture}[
  node distance=1.5cm,
  startstop/.style={rectangle, rounded corners, minimum width=3cm, minimum height=1cm,text centered, draw=black, fill=red!30},
  process/.style={rectangle, minimum width=3cm, minimum height=1cm, text centered, draw=black, fill=blue!30},
  decision/.style={diamond, minimum width=3cm, minimum height=1cm, text centered, draw=black, fill=green!30}
]
  % 节点定义
  \node (start) [startstop] {开始};
  \node (proc1) [process, below of=start] {编写 LaTeX 源码};
  \node (proc2) [process, below of=proc1] {用 latexmk 编译};
  \node (dec1) [decision, below of=proc2] {编译成功?};
  \node (proc3) [process, below of=dec1, yshift=-0.5cm] {用 Skim 预览 PDF};
  \node (stop) [startstop, below of=proc3] {结束};
  \node (proc4) [process, right of=dec1, xshift=3cm] {修改错误};

  % 连线
  \draw [->] (start) -- (proc1);
  \draw [->] (proc1) -- (proc2);
  \draw [->] (proc2) -- (dec1);
  \draw [->] (dec1) -- node[anchor=east] {是} (proc3);
  \draw [->] (dec1) -- node[anchor=south] {否} (proc4);
  \draw [->] (proc4) -- (proc2);
  \draw [->] (proc3) -- (stop);
\end{tikzpicture}

% 4. 参考文献测试(测试 bibtex + 引用)
\section{参考文献测试}
本文引用了爱因斯坦的相对论论文 \cite{einstein1905} 和 LaTeX 学习手册 \cite{latex2e}。

% 生成参考文献列表(按引用顺序排序)
\bibliographystyle{unsrtnat}  % 参考文献样式(自然排序,带编号)
\bibliography{ref}  % 关联前面定义的 ref.bib 数据库

\end{document}
