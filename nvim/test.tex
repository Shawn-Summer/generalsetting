\documentclass{article}
% 核心:中文支持(二选一,推荐 ctex 宏包,简单易用)
\usepackage{ctex}  % 自动处理中文编码、字体,支持 XeLaTeX/LuaLaTeX 编译
% 若需更精细控制字体,可改用 xeCJK 宏包(需手动指定字体):
% \usepackage{xeCJK}
% \setCJKmainfont{SimHei}  % 中文主字体(黑体),需系统安装对应字体

% 原有宏包保留
\usepackage{graphicx}   % 图片插入
\usepackage{amsmath}    % 数学公式
\usepackage{amssymb}    % 数学符号
\usepackage{float}      % 图片浮动位置控制
\usepackage{geometry}   % 页面边距
\geometry{a4paper, margin=1in}  % A4纸,1英寸边距

\title{LaTeX 图片与公式中文示例}  % 中文标题
\author{测试作者}
\date{\today}

\begin{document}

\maketitle

\section{基础数学公式}
LaTeX 支持行内公式和行间公式两种排版方式,中文可以直接嵌入文本。

\subsection{行内公式}
行内公式直接嵌入文本,例如勾股定理:$a^2 + b^2 = c^2$,其中 $a, b$ 为直角边,$c$ 为斜边。

\subsection{行间公式}
行间公式单独成行,适合复杂表达式(如高斯积分):
\[
\int_{-\infty}^{+\infty} e^{-x^2} dx = \sqrt{\pi}
\]
使用 \texttt{align} 环境可以实现多公式对齐(三角函数基本关系):
\begin{align*}
\sin^2 \theta + \cos^2 \theta &= 1 \\
\tan \theta &= \frac{\sin \theta}{\cos \theta}
\end{align*}
上述公式使用了 \texttt{align*} 环境,带星号表示不自动编号。

\section{图片插入示例}
插入图片需要准备图片文件(如 PNG、JPG、PDF 格式),并放在与 tex 文件同一目录下。

\begin{figure}[H]  % [H] 强制固定图片位置
    \centering  % 图片居中
    \includegraphics[width=0.6\textwidth]{example-image}  % 内置测试图片,无需额外文件
    \caption{LaTeX 内置测试图片(中文标题)}  % 中文图片标题
    \label{fig:test}  % 图片标签,用于交叉引用
\end{figure}

如图 \ref{fig:test} 所示,这是一张测试图片。你可以将 \texttt{example-image} 替换为自己的图片文件名(无需后缀,建议英文命名,避免中文路径)。

\section{公式与图片结合示例}
球体体积公式为:
\[
V = \frac{4}{3} \pi r^3
\]
其中 $r$ 为球体半径,对应的球体示意图可参考相关几何教材图片。中文段落可以正常换行、对齐,与公式、图片互不冲突。

\end{document}
